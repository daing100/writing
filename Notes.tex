\documentclass[12pt,a4paper]{report}
\usepackage[utf8]{inputenc}
\usepackage{amsmath}
\usepackage{amssymb}
\usepackage{amsthm}
\usepackage{amsfonts}
\usepackage{makeidx}
\usepackage{graphicx}
\usepackage{tikz}
\usepackage{tikz-feynman}


% If using spacemacs in Vim mode, use ', a' to save and build.

\title{Notes}
\author{Carl-Fredrik Lidgren}

\renewenvironment{\thesection}{\arabic{section}}

\newcommand*{\dd}{\mathop{}\!\mathrm{d}}
\newcommand{\I}[1]{\mathrm{I}_{#1}}
\newcommand{\Res}[2]{\mathrm{Res}_{#1}\!\left( #2 \right)}
\newcommand{\p}{\partial}
\newcommand{\diam}{\mathrm{diam}\,}

\renewcommand\qedsymbol{QED}

\newtheorem{definition}{Definition}

\begin{document}
\maketitle
\tableofcontents
\clearpage

\section{Introduction}
This document is for some general thoughts and notes on mathematics. This may be
transfered to another file in the future.
\subsection{Symbols}
\begin{tabular}{c p{12.5cm}}
  $\forall$ & ``For all'' or ``for each''. Sometimes, if adventurous, ``for every''.\\
  $\exists$ & ``Exists'' or ``there exists''.\\
  $!\exists$ & ``There exists a unique''.\\
  $\therefore$ & ``Therefore'' or ``thus''.\\
  $||$ & ``Such that''. In set builder notation, this will simply be '$|$'.\\
  $\in$ & ``In'' or ``is in''.\\
  $\subset$ & ``Subset'' or ``is a subset of''.\\
  $\Rightarrow$ & ``Implies'' or ``then''.\\
  $\Leftrightarrow$ & ``If and only if''. $p \Leftrightarrow q$ is the same as $p \Rightarrow q,\;q\Rightarrow$.
\end{tabular}

\chapter{Mathematics}
\clearpage
\section{Real Analysis}
Most of the stuff here relates to Walter Rudin's ``Principles of Mathematical
Analysis'', but other stuff might pop up as well.\\
\begin{definition}
  A set $X$ equiped with a distance function is called a metric space. The
  distance function must satisfy the following for every $p,q\in X$:
  \begin{align*}
    &(a)\quad d(p,q) > 0 \text{ if } p\not = q\\
    &(b)\quad d(p,p) = 0\\
    &(c)\quad d(p,q) = d(q,p)\\
    &(d)\quad d(p,q) \leq d(p,r) + d(r,q),\text{ for any } r\in X
  \end{align*}
\end{definition}
A lot of the definitions assume some sort of metric space $X$.
\begin{definition}
  A neighborhood \(N_r(p) = \{ q\in X\;|\;\, d(p,q)<r \}\).
\end{definition}
If you consider $X=\mathbb{R}^2$, then a neighborhood of would be the set of all
points in an open circle of some radius around $p$. With this definition in
mind, we can move on to some more complicated things.
\begin{definition}
  A limit point of the set $E$ is a point \(p\) such that
  \[
    \forall N(p)\,\exists q \not = p,\,q\in N(p),\;q\in E
  \]
\end{definition}
This effectively means that the point $p$ is right on the edge of the set. Of
course, most points of a set are limit points, bui the interesting ones are
generally the ones which are sort of at the edge of the set.
\begin{definition}
  An interior point of the set $E$ is a point $p$ such that \[\exists N(p)
  \subset E\]
\end{definition}
This definition is significantly easier to grasp than the limit point
definition. It means that a point $p$ is an interior point of $E$ if you can form a
circle around it which is inside the set $E$.
\begin{definition}
  A set $E$ is called \textbf{closed} if every limit point of $E$ is a point in $E$.
\end{definition}
It is easy to see how this mimics the intuitive meaning of 'closed', since if
you imagine a closed interval or closed circle, then it will have a border which
is exactly what is being described here by the limit points.
\begin{definition}
  A sequence $\{ p_n \}$ is said to converge to a point $p\in X$ if \[ \forall
    \varepsilon > 0\,\exists N\,||\, n\geq N\Rightarrow d(p_n,p) < \varepsilon \]
  In this case, we write either $p_n\rightarrow p$ or \[ \lim_{n\rightarrow \infty}p_n=p \]
\end{definition}
The reason for this definition should be obvious: if the terms of the sequence
get closer and closer to the point $p$, then the sequence converges to p. As
such, it is natural to take this definition further and define a cauchy
sequence.
\begin{definition}
  A sequence $\{p_n\}$ is said to be a \textbf{cauchy sequence} if\[\forall
    \varepsilon > 0 \,\exists N \,||\, n,m\geq N\Rightarrow d(p_n,p_m) <
    \varepsilon \]
\end{definition}
This definition is a little more difficult to process than the one for a
traditional convergent sequence, but effectively what it means is that a cauchy
sequence is a sequence wherein each term progressively gets closer and closer to
the next one.
\begin{definition}
  If $E\subset X$, and $S=\{d(x, y)\;|\;\, x,y\in E \}$, then we say that $\sup
  S$ is the diameter of $E$, and is written \(\diam E\).
\end{definition}
Given this, we can create an equivalent definition of a cauchy sequence. If
$\{p_n\}$ is a sequence, and $E_N = \{p_N,p_{N+1},p_{N+2},...\}$, then $\{p_n\}$
is a cauchy sequence if and only if \[ \lim_{N\rightarrow\infty}\diam E_N=0 \]

\chapter{Physics}
\clearpage
\section{Nuclear and Particle Physics}
\subsection{Light Energy is Quantized}
The energy of light comes in small packages, given by the formula \[E = hf.\]
A consequence of this is that we can very easily determine the energy released
by a photon when an atom changes energy states. The energy of a hydrogen atom at
\(n=n_0\) is given by \[E = -\frac{13.6}{n^2}.\] As such, we can get the
difference in energy between levels, e.g. \(n_1\rightarrow n_2\):
\[\Delta E = -13.6\left( \frac{1}{n_2^2}-\frac{1}{n_1^2} \right).\]
So, from this, since the energy of the photon released in this interaction is
the difference in energy \(\Delta E\), we can determine the frequence of the
light by the following:
\[
  \Delta E = hf \Rightarrow f = \frac{\Delta E}{h} = \frac{-13.6\left( \frac{1}{n_2^2}-\frac{1}{n_1^2} \right)}{h}
\]
Which gives us the final answer,
\[
  f = \frac{13.6}{h}\left( \frac{1}{n_1^2}-\frac{1}{n_2^2} \right)
\]
Of course, this only holds for the hydrogen atom. The constant \(-13.6\) would be
different for other atoms.
\subsection{Binding Energy}
The binding energy of a nucleon in a particular nucleus is the amount of energy
it would take to remove the nucleon completely. This can be calculated rather
easily by simply taking the difference between the experimentally determined
mass and the theoretically determined mass of the nucleus. For example, given an
atom \(_Z^AK\), we get the mass difference \(\delta\) by the following:
\begin{align*}
  \delta &= m_K - (Zm_e + Zm_p + (A-Z)m_n)\\&= m_K - Z(m_e+m_p) - (A-Z)m_n\\&= m_K -
  Z(m_e+m_p-m_n) + Am_n
\end{align*}
From this, it is trivial to get the energy. Simply apply the energy-mass
equivalence formula,
\[
  E = mc^2 = \delta c^2
\]
\subsection{Feynman Diagrams}
Feynman diagrams are a convenient way of representing interactions between
particles. While you could, for example, write out a long chain of characters
representing the interaction, you could also just have a nice diagram showing it
just as accurately, if not more so.
\end{document}
